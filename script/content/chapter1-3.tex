\section{Einstellungen}\index{Einstellungen}\label{sec:einstellungen}

In der Datei \verb|inc/settings/settings.tex| finden sich alle wichtigen Einstellungen, die von dieser Vorlage angeboten werden, um das Layout oder Verhalten zu bestimmen. Unge�bten \LaTeX-Anwendern wird empfohlen, nur diese Einstellungen zu variieren und f�r �nderungen innerhalb der Vorlage Hilfe zu konsultieren, um unerw�nschte Effekte zu vermeiden. Manche Schalter nehmen lediglich die Werte \verb|0| und \verb|1| an\footnote{Tats�chlich kann man ihnen beliebige Werte zuweisen, lediglich der Wert \texttt{1} aktiviert den Schalter jedoch.}. Der Wert \verb|1| aktiviert, der Wert \verb|0| deaktiviert den entsprechenden Schalter.

An dieser Stelle wollen wir im Detail auf die einzelnen Schalter und Optionen in der eben erw�hnten Datei eingehen:
\begin{enumerate}
	\item \verb|\docAutor|, \verb|\docTitel|, \verb|\docUntertitel|, \verb|\docDozent|, \verb|\docJahr|, \verb|\docUniversitaet| -- mit diesen Befehlen wird die Ausgabe auf der Titelseite\footnote{In der Standarddatei f�r die Rechtshinweise werden diese Befehle ebenfalls verwendet.} gesteuert und automatisch angepasst.
	\item \verb|\docTitelZitat|, \verb|\docTitelZitatName| -- diese beiden Befehle betreffen ebenfalls die Titelseite. Es kann ein Zitat angegeben werden, das standardm��ig unter dem Titel erscheint. Aufgrund der unterschiedlichen Formatierung wird der Name der zitierten Person getrennt gespeichert. Der Befehl darf jedoch auf keinen Fall gel�scht werden, sondern muss einfach mit einem leeren Inhalt definiert werden.
	\item \verb|\useIndex| -- mit diesem Schalter kann festgelegt werden, ob ein Stichwortverzeichnis ausgegeben werden soll.
	\item\label{var:usethumbs} \verb|\useThumbs| -- dieser Schalter aktiviert kleine K�stchen am rechten Seitenrand, die bei gedruckter Form helfen, bequem das richtige Kapitel aufzuschlagen. Es gilt jedoch zu beachten, dass nur wenige Drucker den Rand wirklich bedrucken k�nnen.
	\item \verb|\useRoman| -- Mit dieser Einstellung werden Kapitel, die mit \verb|\chapter| er�ffnet werden, mit r�mischen statt arabischen Ziffern nummeriert.
	\item \verb|\usrBCOR| -- da beim Binden eines Buches Teile des inneren Randes augenscheinlich verschwinden wird die Bindekorrektur (BCOR, engl. \emph{binding correction}) eingef�hrt.
	\item\label{var:usrmatter} \verb|\usrmatter| -- ist diese Option aktiviert, so werden die Seiten vor dem eigentlichen Inhalt des Buches separat nummeriert und in r�mischen Zahlen ausgegeben. Die Nummerierung beginnt ab dem eigentlichen Inhalt dann neu.
	\item \verb|\usroptsqrt| -- mit diesem standardm��ig aktiviertem Schalter wird die Form des Wurzelsymbols angepasst, so dass ein schlie�ender Strich eingef�gt wird: $\sqrt{x}$ versus $\oldsqrt{x}$ % Anmerkung: Der Befehl \oldsqrt wird hier nur zur Demonstrationszwecken verwendet; von einer Anwendung im Dokument ist dringend abzuraten!
	\item\label{var:usrnscmd} \verb|\usrnscmd| -- hier wird der Befehl eingestellt, der f�r die Zahlenraum-Makros \verb|\A|, \ldots, \verb|\Z| verwendet wird (siehe \ref{sec:makros}). Eine g�ngige Alternative ist \verb|\mathbb|, sch�ner w�re jedoch \verb|\mathds|. Als Standard wird eine simple Fettschrift gesetzt, da dies auch der traditionellen Form entspricht.
\end{enumerate}
