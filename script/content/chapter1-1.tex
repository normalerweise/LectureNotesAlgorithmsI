\chapter{Benutzung der Skriptvorlage}

\begin{beschreibung}
In diesem Kapitel wollen wir die vielf�ltigen M�glichkeiten, die von dieser Vorlage bereitgestellt werden, kennenlernen und uns so einen Eindruck �ber ihre M�glichkeiten verschaffen. Wir werden sehen, wie diese Vorlage korrekt eingestellt und gehandhabt werden kann, um sch�ne Ergebnisse zu erzielen.
\end{beschreibung}

\section{Installation}\index{Installation}

Da man die Vorlage gegebenenfalls mehrfach verwenden m�chte und "`Skriptvorlage"' ein eher weniger aussagekr�ftiger Projekttitel ist, sollte darauf verzichtet werden, direkt in die Vorlagedateien zu schreiben.

Stattdessen sollte zun�chst ein neues Projekt angelegt werden\footnote{Falls die verwendete Editor-Umgebung dieses Feature zur Verf�gung stellt.}. In die Hauptdatei wird dann per "`Copy \& Paste"' der Inhalt der \verb|Skriptvorlage.tex| kopiert. Auf diese Weise legt man sich eine neue Kopie der Vorlage an und kann sie gleich entsprechend benennen. Die �brigen Dateien k�nnen dann einfach in das neue Verzeichnis kopiert werden. Nach diesen einfachen Schritten ist die Vorlage gewisserma�en "`installiert"' und als neues Projekt angelegt worden.

Der Inhalt sollte nun kapitelweise (siehe \ref{subsec:modular}) in Teildokumente aufgeteilt werden, die im Ordner \verb|content/| untergebracht werden k�nnen. In der Hauptdatei muss dann an der entsprechenden, gekennzeichneten Stellen f�r jede Datei ein \verb|\include|-Befehl gesetzt werden, um die Datei einzubinden. Es gilt zu beachten, dass beim Einbinden einer neuen Datei stets eine neue (rechte) Seite begonnen wird\footnote{Da der Befehl \texttt{\textbackslash cleardoublepage} aufgerufen wird, werden insbesondere auch alle noch nicht ausgegebenen Figuren ausgegeben.}.