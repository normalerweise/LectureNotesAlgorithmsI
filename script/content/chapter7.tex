\chapter{Approximation Algorithms}
An algorithm for an optimization problem that yields a ``nearly''
optimal solution is called an approximation algorithm.
$0<C_{n}*$ cost of an optimal solution for an input of size
$n$ (e.g. length of a shortest path). $0<C_{n}$ cost of a solution
produced by an approximation algorithm. If 
$max(\frac{C_{n}}{C_{n}*},\frac{C_{n}*}{C_{n}})=\rho (n)$ then
the algorithm has an approximation ratio of $\rho (n)$.\\
(Remark: $\rho (n) \ge 1$)\\
$\rho (n) = 1$: optimal, $\rho(n) >> 1$: bad approximation\\
Trade-off: Cost of the approximation algorithm vs. quality of approximation.\\
An approximation scheme for an approximation problem is an 
algorithm with 2 inputs: $Problem + \epsilon>0$ such that
the algorithm is a $(1 + \epsilon)$ - approximation algorithm.\\
The approximation scheme is a polynomial time approximation
scheme if for each $\epsilon > 0$ the induced algorithm runs in polynomial time in $n$.
It is fully polynomial if it is polynomial in $n$ and in $1/\epsilon$.
\begin{example}
$O(n^(\frac{2}{3}))$, if $\epsilon' = \frac{3}{4}$ $O(n^(\frac{8}{\epsilon}))$ cost, i.e.
the costs are rising but not by a constant factor.
\end{example}
\begin{example}
$O((1/\epsilon)^2 * n^3)$ fully polynomial. $\epsilon = \frac{\epsilon}{4}
O(16*(\frac{1}{\epsilon}^2 * n^3))$.
\end{example}
\section{Heuristic Approaches}
\emph{Vertex Cover}: Given an undirected graph $G=(V,E)$ without parallel edges and without loops,
we look for a minimal set $V'$ of nodes such that each edge has at least one endpoint in $V'$.
\begin{example*}
\includegraphics[width=\textwidth]{diagrams/Chapter7_Example1.pdf}
\end{example*}
\begin{algorithm}
\begin{algorithmic}[1]
\State $C \gets \emptyset$
\State $E' \gets E$
\While{$E' \neq \emptyset$}
	\State Choose an edge $\{u,v\}$ in $E'$.
	\State $C \gets C \cup \{u,v\}$
	\State Remove all edges from $E'$ with the end point $u$ or $v$.
\EndWhile
\end{algorithmic}
\caption{APPROX-VERTEX-COVER($G=(V,E)$)}
\end{algorithm}

\section{Probabilistic Analysis and randomized algorithms}

\textbf{Assistant search}
\begin{enumerate}
  \item[1] best = 0
  \item[2] for i = 1 .... n do\\
\noindent\hspace*{10mm}	interview candidate i\\
\noindent\hspace*{10mm}	if candidat i is better than best\\
\noindent\hspace*{20mm}		then best:= i\\
\noindent\hspace*{20mm}		employ candidate i (fire previous candidate)\\
\end{enumerate}

\underline{Analysis} of the cost that arise from interviews and hiring.\\
cost of an interview $c{_I}(low)$\\
cost of hiring: $c{_E}(high)$\\
Cost that arise in the worst case $O(n* c{_I} +  m * c{_E}$ ) if m people are hired 
and \\
$O(n * c{_I} +  n *  c{_E} )$ if n people are hired \\
worst case: if candidates show up in the order: worst first, \ldots best last

\textbf{Indicator Variable}
Let S be a set and $A \subseteq S$ and execute the indicator variable.\\

$ X(s)=\left\{\begin{array}{cl} 1, if s\in A \\ 0, else \end{array}\right .$

$E[x{_1}] = \sum\limits_{x} x Pr(X{_A}=x) = Pr(A)$

\textbf{Example}
Throw a coin n times.
$ X(s)=\left\{\begin{array}{cl} 1, \mbox{ if in the i-th throwing head is shown} \\ 0, \mbox{  otherwise } \end{array}\right .$

$S = {(a{_1},\ldots,a{_n}) : a{_i} \in {H,N}}$,
Let X be the random variable that describes how often ``H'' appears when you throw n-times.

$X:\{H,N\}^n \rightarrow \mathbb{N}$
$X=\sum x{_i}$

The expected number of ``H''\\

$E[X] = E[\sum x{_i}] = \sum\limits_{i=1}^n E[x{_i}] = \sum\limits_{i=1}^n \frac{1}{2} = \frac{n}{2}$

\textbf{Example Hiring:}\\

$ X{_i} = \left\{\begin{array}{cl} 1, \mbox{ if the candidate is hired} \\ 0, \mbox{  else } \end{array}\right .$

$X= \sum\limits_{i=1}^n x{_i}$ describs how many candidates are hired\\

$E[X] =  E[\sum\limits_{i=1}^n x{_i}] = \sum\limits_{i=1}^n E[x{_i}]$\\

$E[X{_i}] = $ Probability that the i-th candidate is hired.


Candidate i is hired if this person is better than the first i-1 candidates. If the order in which people are chosen is random, each of the first i candidates is the best with same probability.
So $E[x{_i}] =\frac{1}{i}$\\
$E[X] =  \sum\limits_{i=1}^n \frac{1}{i} = \ln n + O(1)$ 


\begin{lemma}
If the order in which the candidates appear is random then the average cost of finding a suitable employee is
$ O( c_E *  \ln n + C_I * n)$
\end{lemma}

\textbf{Question:}
How can we know that the order is random ?\\

\subsection{Randomized Algorithms}
In order to guarantee randomness of the input, we randomize it. 

\begin{example}{Hiring}
Given the list of the n candidates, use random numbers to select a candidate for an interview.
$RANDOM(a,b)$ produces an integer z with  $a \leq z \leq b$ where all numbers in this interval are equally probable. 
$RANDOM(0,1)$ produces 0 or 1 with equal probability $\frac{1}{2}$
$RANDOM(3,7)$ produces 4 with probability $\frac{1}{5}$ \\
\end{example}
  
\textbf{Randomized-Assistant Search}
\begin{enumerate}[label={\arabic*.}]
 \item Permute the list of candidates using a  random number generator
 \item best = 0
 \item ... as before
\end{enumerate}

\begin{lemma}
The expected cost for hiring using the randomized assistant search algorithm is \\
$O( c_E * \ln n + C_I * n)$
\end{lemma}
How to randomize an in input of n elements [A[1...n]]? \\
PERMUTE-BY-SORTING(A) \\
1.	n = length(A) \\
2.	for i = 1...n do \\
3.	P[i] = RANDOM [1, $n^3$] (something arbitrary bigger than n) \\
4.	Sort the array A using P as sort keys \\
5.	return A \\
\newline
RANDOMIZE-IN-PLACE(A) \\
1.	n = length(A) \\
2.	for i= 1...n do \\
3.	  swap A[i] $\Leftrightarrow$  A[RANDOM(i,n)] \\
>>>>>>> 537543df52b95855cfa2f207fc7482bd87d8b8b6

