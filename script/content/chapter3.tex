\chapter{(Network) Flow Problems}

\begin{descr}
    TODO
\end{descr}

\section{Network Flow Problems}\index{Network Flow Problems}
\begin{example}
Example: Oil field + transportation

\end{example}

\begin{definition}
A network N consists of 
\begin{enumerate}
\item A finite directed graph $G=(V,E)$ without loops and parallel edges
\item a function $c: E -> \mathbb{R}^{+}$, which assigns a capacity to each edge
\item two designated nodes s and t, called \deftxt{source} and \deftxt{sink}
\end{enumerate}
Short: $N = (G, c, \{s, t\})$
\end{definition}

\begin{definition}
Let $N = (G, c, \{s, t\})$ be a network. A flow function on N is a function $f: E -> \mathbb{R}$ such that 
\begin{itemize}
\item $ 0 \le f(e) \le c(e), \forall e \in E $
\item $ \alpha(v) := \{e: endpoint of e is v\}, v \in V$ \\
$\beta(v) := \{e: startpoint of e is v\}, v \in V$

\end{itemize}
\end{definition}