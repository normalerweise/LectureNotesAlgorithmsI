\section{Verf�gbare Umgebungen}\index{Umgebungen}

Nachdem wir nun die Makros kennen, die von der Vorlage bereitgestellt werden, kommen wir zu den verf�gbaren Umgebungen. Ihre konsequente Verwendung wird f�r einen typographisch guten Stil dringend empfohlen:
\begin{enumerate}
	\item Definitionen k�nnen mit der \verb|definition|-Umgebung erstellt werden.\\
	\verb|\begin{definition} \ldots \end{definition}| liefert:
	\begin{definition}
	\ldots
	\end{definition}
	Die Nummerierung erfolgt automatisch entsprechend des aktuellen Kapitels. Alle Umgebungen teilen sich einen "`Z�hler"', so dass jede Nummer einzigartig ist.
	\item F�r S�tze, Korollare, Lemmata und Propositionen gibt es entsprechende Umgebungen verm�ge \verb|satz|, \verb|korollar|, \verb|lemma| und \verb|proposition|. Analog zu der \verb|definition|-Umgebung liefert dies beispielsweise:
	\begin{satz}[Satz von \ldots]
	\ldots
	\end{satz}
	\item F�r Beispiele stehen die Umgebung \verb|beispiel| bzw. \verb|beispiel*| f�r unnummerierte Beispiele zur Verf�gung:
	\begin{beispiel}
	\ldots
	\end{beispiel}
	Endet ein Beispiel beispielsweise in einer Liste oder in einer abgesetzten Gleichung, so kann mit \verb|\qedhere| der Endmarker manuell an die entsprechende Stelle gesetzt werden.
\end{enumerate}
All diese Umgebungen erlauben zudem einen optionalen Parameter in eckigen Klammern -- also z. B. \verb|\begin{satz}[Bezeichnung]| --, welcher f�r den Titel der Umgebung gew�hlt wird. Je nach Umgebung kann dies eine Bezeichnung des Theorems oder der zu definierende Begriff sein. Ein Beispiel hierf�r wurde eben schon gezeigt.

\begin{information}
Die Umgebungen sichern sich mit Hilfe des \verb|needspace|-Pakets bereits Platz, dennoch kann es unter Umst�nden zu leichten Darstellungsfehlern kommen, bei denen oben oder unten der Leerraum zum farbigen Hintergrund fehlt. Dieser Fehler l�sst sich bisher leider nur durch redaktionelle Eingriffe im Einzelfall beheben.
\end{information}

\begin{enumerate}[resume]
	\item F�r Beweise steht die \verb|beweis|-Umgebung zur Verf�gung. Ihr Aussehen ist schlicht gehalten. Das \verb|\qed|-Symbol wird automatisch eingef�gt; sollte der Beweis jedoch mit einer abgesetzten Gleichung enden, so sollte der Befehl \verb|\qedhere| innerhalb dieser Gleichung verwendet werden, um das Symbol korrekt zu positionieren.
	\begin{beweis}
	Hier steht ein Beweis.
	\end{beweis}
	\item Die Umgebung \verb|information| dient f�r informative Hinweise neben des eigentlichen Themas. So k�nnen hiermit interessante Ausblicke in andere Anwendungen, Fachgebiete oder auch in die Geschichte zu einem Thema gegeben werden.
	\begin{information}
	Diese Umgebung sollte gezielt eingesetzt werden, um den Lesestoff zwar aufzulockern, jedoch nicht vom eigentlichen Thema abzulenken. Der Inhalt sollte informativ, aber kurzgehalten sein.
	\end{information}
	\item Die \verb|\beschreibung|-Umgebung sollte zu Beginn eines jeden Kapitels -- also nach dem Aufruf des \verb|\chapter|-Befehls -- verwendet werden, um einen einheitlichen Stil im Buch zu erhalten (oder, andernfalls, gar nicht verwendet werden). Sie dient dazu, den Inhalt des kommenden Kapitels kurz zusammenzufassen und so einen Einblick in die bevorstehende Thematik zu gew�hren. Ein Beispiel f�r diese Umgebung findet sich zu Beginn dieses Kapitels.
\end{enumerate}
