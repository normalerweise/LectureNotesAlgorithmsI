\section{Verf�gbare Makros}\index{Makros}\label{sec:makros}

Diese Vorlage bietet standardm��ig bereits viele vordefinierte Makros und Umgebungen. Die Definitionen der Makros werden in der Datei \verb|inc/settings/abbreviations.tex| verwaltet. Je nach pers�nlichen Pr�ferenzen und der Notwendigkeit kann diese vordefinierte Liste ver�ndert oder erweitert werden.
\begin{information}
Makros sind ein wichtiges Werkzeug, um einen konsistenten Stil zu gew�hrleisten. Sie erleichtern zudem die Arbeit, da etwaige �nderungen nur einmalig und nicht m�hsam im gesamten Dokument durchgef�hrt werden m�ssen.
\end{information}

An dieser Stelle wollen wir alle verf�gbaren Makros kennenlernen:
\begin{enumerate}
	\item Die Befehle \verb|\A|, \ldots, \verb|\Z| liefern Buchstaben f�r Zahlenr�ume, lediglich \verb|\H|, \verb|\L|, \verb|\O| und \verb|\P| stehen nicht standardm��ig zur Verf�gung. Der Befehl \verb|\I| ist ebenfalls andersweitig belegt (siehe \ref{makro:im}). Es existieren nur Gro�buchstaben mit diesen Befehlen. Die Makros verwenden intern den Befehl \verb|\usrnscmd|\footnote{Siehe \ref{sec:einstellungen}, \ref{var:usrnscmd}.}, der standardm��ig auf \verb|\textbf| eingestellt ist.
	\begin{align*}
	\A, \B, \C, \D, \E, \F, \G, \J, \K, \M, \N, \Q, \R, \S, \T, \U, \V, \W, \X, \Y, \Z.
	\end{align*}
	Weit verbreitet sind auch die Befehle \verb|\mathbb| und \verb|\mathds| f�r doppelgestrichene Buchstaben. Historisch gesehen wurden diese aber nur eingef�hrt, da Fettschrift an der Tafel schwer umzusetzen war, weswegen diese Varianten eigentlich nicht verwendet werden sollten. Da sie dennoch sehr verbreitet sind werden sie unterst�tzt. Es wird empfohlen, in diesem Fall \verb|\mathds| zu verwenden, da die Doppelstriche an der korrekten Position sind und die Schriftzeichen die richtige Gr��e haben.
	\item Analog liefert \verb|\sA|, \ldots, \verb|\sZ| kalligrafisch geschwungene Buchstaben. Auch hier gibt es nur Gro�buchstaben, es existieren hierbei aber sogar alle Buchstaben. Intern verwendet wird der Befehl \verb|\mathcal|.
	\begin{align*}
	\sA, \sB, \sC, \sD, \sE, \sF, \sG, \sH, \sI, \sJ, \sK, \sL, \sM, \sN, \sO, \sP, \sQ, \sR, \sS, \sT, \sU, \sV, \sW, \sX, \sY, \sZ.
	\end{align*}
	\item Mit dem Befehl \verb|\deftxt{\ldots}| k�nnen definierte Begriffe hervorgehoben werden.\\
	\verb|Wir nennen dies einen \deftxt{Begriff}| liefert:
	\begin{align*}
	\text{Wir nennen dies einen \deftxt{Begriff}}
	\end{align*}
	\item\label{makro:im} F�r die imagin�re Einheit $\I$ existiert das Makro \verb|\I|.
	\item F�r das Differential $\dd$ in Integralen steht das Makro \verb|\dd| zur Verf�gung.
	\item F�r das h�ufig verwendete $\e$ gibt es den Befehl \verb|\e|.
	\item Mit \verb|\id| erh�lt man den Identit�tsoperator $\id$.
	\item F�r die Indikatorfunktion kann \verb|\ind_A(x)| verwendet werden und liefert dann $\ind_A(x)$.
	\item Eine Norm kann mit \verb|\norm{x}| geschrieben werden und liefert $\norm{x}$. Die Normstriche passen sich in der Gr��e automatisch an.
	\item Der Realteil $\Re z$ und der Imagin�rteil $\Im z$ komplexer Zahlen lassen sich mit \verb|\Re z| und \verb|\Im z| notieren, verhalten sich also genauso wie die g�ngigen Operatoren wie z. B. \verb|\sin x| f�r $\sin x$.
	\item F�r Notationen der Stochastik existieren \verb|\Var| f�r die Varianz $\Var$, \verb|\Cov| f�r die Kovarianz $\Cov$ und \verb|\Pois| f�r die Poisson-Verteilung $\Pois$.
\end{enumerate}